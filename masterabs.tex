% pLaTeX でタイプセットする
% platex masterabs && biber masterabs && platex masterabs && platex masterabs && dvipdfmx masterabs.dvi

%%%%%%%%%%%%%%%%%%%%%%%%%%%%%%%%%%%%%%%%%%%%%%%%%%%%%%%%%%%%%%%%
%%%%%%%  Example: extended abstract for master thesis
%%%%%%%  version 1.0
%%%%%%%  file name: template.tex
%%%%%%%%%%%%%%%%%%%%%%%%%%%%%%%%%%%%%%%%%%%%%%%%%%%%%%%%%%%%%%%%
%--------------- start preamble -------------------------------
\documentclass[a4paper]{jarticle} % 10pt fonts, default fonts
%\documentclass[a4paper,11pt]{jarticle} % 11pt fonts
%\documentclass[a4paper,12pt]{jarticle} % 12pt fonts
%--------------------------------------------------------------
\usepackage{masterabs} % 修士論文アブストラクトのスタイルファイル
%--------------------------------------------------------------
\usepackage{amsmath,amsthm,mathrsfs} % amslatex モードの指定
\usepackage{amsfonts,amssymb,txfonts} % amsfonts の指定
% \usepackage{graphicx} % 図の挿入の指定 (\includegraphicsなど)
%--------------------------------------------------------------
% \columnseprule = 0.4pt % two columnの真ん中に縦線を引く
%--------   英文の場合: 表,図、参考文献を英語に変更 ----------------
\initenglish % 本文が英文の場合は % を取る(表=>Tab., 図=>Fig.など)
%--------------------------------------------------------------------
%

\theoremstyle{definition}
\newtheorem{thm}{Theorem}[section]
\newtheorem{dfn}[thm]{Definition}
\newtheorem{eg}[thm]{Example}
\newtheorem{lem}[thm]{Lemma}
\newtheorem{cor}[thm]{Corollary}
\theoremstyle{remark}
\newtheorem{rem}[thm]{Remark}

\DeclareMathOperator{\rank}{rank}
\DeclareMathOperator{\NS}{NS}
\DeclareMathOperator{\Tr}{Tr}
\DeclareMathOperator{\chara}{char}
\DeclareMathOperator{\Spec}{Spec}
\DeclareMathOperator{\ch}{ch}
\newcommand{\Neron}{N\'eron}

\usepackage[T1]{fontenc}

\usepackage[sorting=nyt,date=year,isbn=false,doi=false,url=false,giveninits]{biblatex} % biblatexを使用するためのパッケージ
\addbibresource{references.bib}

%-------------- end preamble ----------------------------------
%
%%%%%%    TEXT START    %%%%%%

\begin{document}
%
%-------------- two column -------------------------
% \twocolumn[ % two column の場合は,先頭の % を取る
%---------------------------------------------------
%
%---------------------------------------------------------------------------
\no_tlfnmark % タイトルの最後にfootnote markを付けない場合は,先頭の % を取る
%---------------------------------------------------------------------------
%%% タイトルが 1行 \title{タイトル}を使う
%%% タイトルが 2 行にわたるときは \2ltitle{1行目}{2行目}を使う
%---------------------------------------------------------------
% \title{On the Mordell-Weil groups of elliptic surfaces associated with Frey curves of degree two} % 1 行用
%
\2ltitle{On the Mordell-Weil groups of elliptic surfaces}{associated with Frey curves of degree two} % 2 行用
%
%-------------------------------------------
% 日本語指導教員,著者名など
%-------------------------------------------
\begin{preliminary}
\profname{栗原将人教授}       %% 指導教員の名前 + 講師,准教授,教授
\name{82313206}{八木颯仁} %% 学籍番号, 著者名
\end{preliminary}
%
%---------- two column ----------------------
% ]% two columnの場合は,先頭の % を取る
%--------------------------------------------
%
%------- footnote に英文のタイトルを記述したいとき ----------------
% \etitle{On the Mordell-Weil groups of elliptic surfaces associated with Frey curves of degree two}
%----------------------------------------------------------------
%
\init_fnmark % 脚注マークの初期化(アラビア数字に変更)
%%%%%%%%%%%%%%%%%%%%%%%%%%%%  本文 %%%%%%%%%%%%%%%%%%%%%%%%%%%%%

\section{Introduction}
An elliptic curve is a smooth projective curve of genus $1$.
For points on an elliptic curve we can define an addition law, which makes the set of points on an elliptic curve into an abelian group with the identity element being the point at infinity.
For an elliptic curve $E$ defined over a field $K$, the Mordell-Weil group $E(K)$ is a group consisting of all K-rational points on $E$.

\begin{thm}{(Mordell's Theorem)}
    \label{thm:mordell}
    Let $E$ be an elliptic curve defined over a number field $K$.
    Then the Mordell-Weil group $E(K)$ is a finitely generated abelian group.
\end{thm}
By the structure theorem of finite abelian groups, the Mordell-Weil group can be decomposed into a free part and a torsion part:
\begin{equation*}
    E(K) \cong \mathbb{Z}^{\oplus r} \oplus E(K)_{\text{tors}}
\end{equation*}
where $r$ is the rank of the Mordell-Weil group and $E(K)_{\text{tors}}$ is the torsion subgroup of $E(K)$.
The Mordell-Weil group is an important object in the study of elliptic curves.
Especially, the rank of the Mordell-Weil group is important and difficult to determine.

In this paper, we consider elliptic curves in the form of the Frey curves for $n=2$.
In other words, let $(a,b,c)\in\mathbb{Z}^3$ be a Pythagorean triple and consider the elliptic curve defined by the Weierstrass equation
\begin{equation}
    \label{eq:2frey}
    y^{2} = x(x - a^{2})(x + b^{2}).
\end{equation}

We can parameterize Pythagorean triples $(a,b,c)$ by $m,n \in \mathbb{Z}$ with $(m,n)=1$ as $(a,b,c) = (2mn, m^{2} - n^{2}, m^{2} + n^{2})$.
Then the equation \eqref{eq:2frey} can be written as $y^{2} = x(x - 4m^2n^2)(x + (m^{2} - n^2)^{2})$.
We replace $x,y$ by $n^2x, n^3y$ and put $s = m/n$.
Then we get an elliptic curve
\begin{equation}
    \label{eq:E_{1,s}}
    E_{1,s}: y^{2} = x(x - 4s^{2})(x + (s^{2} - 1)^{2}).
\end{equation}
We consider $E_{1,s}$ as an elliptic curve over a function field $\overline{\mathbb{Q}}(s)$.
We associate an elliptic surface $\mathcal{E}_{1,s} \to \mathbb{P}^1$ to $E_{1,s}$.


\section{Main Theorem}

\begin{thm}
    \label{thm:E_{1,s}}
    The Mordell-Weil group of $E_{1,s}$ over $\overline{\mathbb{Q}}(s)$ satisfies
    \begin{equation*}
        E_{1,s}(\overline{\mathbb{Q}}(s)) \cong \mathbb{Z} / 4 \mathbb{Z} \oplus \mathbb{Z} / 4 \mathbb{Z},
    \end{equation*}
    especially the rank is $0$. The torsion subgroup is generated by
    \begin{align*}
        T_1 & := (2s(s+1)^2, 2s(s+1)^2(s^2+1)),                              \\
        T_2 & := (2 \sqrt{-1} s(s^2-1),2 \sqrt{-1} s(s+\sqrt{-1})^2(s^2-1)).
    \end{align*}
\end{thm}

The following is our main result.
\begin{thm}
    \label{thm:E_{2,t}}
    The Mordell-Weil group of $E_{2,t}$ over $\overline{\mathbb{Q}}(t)$ satisfies
    \begin{equation*}
        E_{2,t}(\overline{\mathbb{Q}}(t)) \cong \mathbb{Z} \oplus \mathbb{Z} / 4 \mathbb{Z} \oplus \mathbb{Z} / 4 \mathbb{Z},
    \end{equation*}
    especially the rank is $1$.
    We denote $s = \frac{2t}{t^{2} - 3}$. The torsion subgroup is generated by $T_1$ and $T_2$ in Theorem~\ref{thm:E_{1,s}} and the free part is generated by
    \begin{equation*}
        \left(s^{2} - 1, \sqrt{-1} s(s^{2} - 1) \frac{t^{2} + 3}{t^{2} - 3} \right).
    \end{equation*}
\end{thm}
The important point is that we prove that the generic rank of $E_{2,t}$ is exactly $1$, not only the existence of a point of infinite order.
Our proof is based on the method of Naskręcki in \cite{ref:naskrecki2013}.


\section{Preliminaries}

In order to get the lower bound of the rank of the Mordell-Weil group, finding points of infinite order is enough.
It is quite difficult to get a good upper bound of the rank.
The following theorem behaves a key role in the proof of the main theorem.
\begin{thm}{(Shioda-Tate formula, \cite[Corollary 5.3]{ref:shioda1990})}
    \label{thm:shioda}
    Let $\mathcal{E} \to C$ be an elliptic surface over a smooth projective curve $C$ over an algebraically closed field $k$.
    Let $R \subset C$ be the set of points where the special fiber of $\mathcal{E}$ is singular.
    For each $v \in R$, let $m_{v}$ be the number of components of the special fiber of $\mathcal{E}$ at $v$.
    Let $\rho(\mathcal{E})$ denote the rank of the \Neron-Severi group of $\mathcal{E}$.
    Then, we have
    \begin{equation*}
        \rho (\mathcal{E}) = 2 + \sum_{v \in R} (m_{v} - 1) + \rank(E(k(C))).
    \end{equation*}
\end{thm}

%%%%%%%%%%%%% 参考文献 %%%%%%%%%%%
% \begin{thebibliography}{99}
%     \bibitem{ref:naskrecki2013} Bartosz Naskręcki, Mordell-Weil ranks of families of elliptic curves associated to Pythagorean triples, Acta Arithmetica, 160 (2013), no. 2, 159-183.
    
% \end{thebibliography}
\printbibliography

\end{document}