%! TEX root = ../main.tex
\documentclass[main]{subfiles}

\begin{document}
\renewcommand{\abstractname}{Thesis Abstract}
\begin{abstract}
    % 400 words 程度
    An elliptic curve is a smooth projective curve of genus $1$.
    On points on an elliptic curve we can define an addition law, which makes the set of points on an elliptic curve into an abelian group with the identity element being the point at infinity.
    For an elliptic curve $E$ defined over a field $K$, the Mordell-Weil group $E(K)$ is a group consisting of all $K$-rational points on $E$.
    The Mordell-Weil theorem states that the Mordell-Weil group is a finitely generated abelian group.
    The Mordell-Weil group is an important object in the study of elliptic curves.
    Especially, the rank of the Mordell-Weil group is important and difficult to determine in general.

    For Pythagorean triples $(a,b,c) \in \mathbb{Z}^{3}$, namely integers satisfies $a^{2} + b^{2} = c^{2}$, we study a family of elliptic curves $y^2 = x (x - a^2)(x+b^2)$.
    This is the $n=2$ case of the Frey curve.
    Using the parameterization of Pythagorean triples $(a,b,c) = (2mn, m^{2} - n^{2}, m^{2} + n^{2})$ by $m,n \in \mathbb{Q}$ with $(m,n)=1$ and putting $s = m/n$, we can one-parameterize Frey curves of degree two, and consider the family as an elliptic curve
    \begin{equation*}
        E_{1,s}: y^{2} = x(x - 4s^{2})(x + (s^{2} - 1)^{2})
    \end{equation*}
    over a function field $\overline{\mathbb{Q}}(s)$.
    It is known that the generic rank of the Mordell-Weil group of $E_{1,s}$ over $\overline{\mathbb{Q}}(s)$ is $0$.
    We first determine the torsion subgroup of the Mordell-Weil group of $E_{1,s}$.
    Also, we found an infinite subfamily of $E_{1,s}$ whose Mordell-Weil rank over $\overline{\mathbb{Q}}(s)$ is $1$, which means that there are infinitely many $s \in \overline{\mathbb{Q}}$ such that the Mordell-Weil group of $E_{1,s}$ has positive rank over $\overline{\mathbb{Q}}$.

    We use the theory of elliptic surfaces to prove it.
    Each elliptic curve over a function field corresponds to an elliptic surface.
    The Shioda-Tate formula gives the relation between the Mordell-Weil rank and the \Neron-Severi rank of the corresponding elliptic surface.

    We determine the types of special fibers of the elliptic surfaces and an upper bound of the rank of the \Neron-Severi group by Tate's algorithm.
    Using this we can prove that the generic rank of the subfamily is no more than $2$.
    However, the upper bound is not sharp.
    In order to prove that the generic rank is exactly $1$, we calculate the characteristic polynomial of the action of the Frobenius automorphism on the second $l$-adic \'etale cohomology group using Lefschetz fixed point theorem, and get the sharp upper bound of the rank of the \Neron-Severi group.
    % 447 words
\end{abstract}
\end{document}