%! TEX root = ../main.tex
\documentclass[main]{subfiles}

\begin{document}
\renewcommand{\abstractname}{Thesis Abstract}
\begin{abstract}
    % 400 words 程度
    For Pythagorean triples $(a,b,c) \in \mathbb{Z}^{3}$, we study a family of elliptic curves $y^2 = x (x - a^2)(x+b^2)$.
    This is the family of the Frey curves of degree $2$.
    Using the parameterization of Pythagorean triples $(a,b,c) = (2mn, m^{2} - n^{2}, m^{2} + n^{2})$ by $m,n \in \mathbb{Q}$ with $(m,n)=1$ and putting $s = m/n$, we can one-parameterize Frey curves of degree two, and consider the family as an elliptic curve
    \begin{equation*}
        E_{1,s}: y^{2} = x(x - 4s^{2})(x + (s^{2} - 1)^{2})
    \end{equation*}
    over a function field $\overline{\mathbb{Q}}(s)$.
    It is known that the generic rank of the Mordell-Weil group of $E_{1,s}$ over $\overline{\mathbb{Q}}(s)$ is $0$.
    We found an infinite subfamily of $E_{1,s}$ whose Mordell-Weil rank over $\overline{\mathbb{Q}}(s)$ is $1$, which means that there are infinitely many $s \in \overline{\mathbb{Q}}$ such that the Mordell-Weil group of $E_{1,s}$ has positive rank over $\overline{\mathbb{Q}}$.
    % \textcolor{red}{TODO: $\mathbb{Q}(s)$上でランク正の無限族じゃないと,Frey curve が無限個とは言い難い}

    We use the theory of elliptic surfaces to prove it.
    Each elliptic curve over a function field corresponds to an elliptic surface.
    The Shioda-Tate formula gives the relation between the Mordell-Weil rank and the \Neron-Severi rank of elliptic surfaces.
    
    We determine the types of special fibers of the elliptic surfaces and an upper bound of the rank of the \Neron-Severi group.
    Using this we can prove that the generic rank of the subfamily is no more than $2$.
    However, the upper bound is not sharp.
    In order to prove that the generic rank is exactly $1$, we calculate the characteristic polynomial of the action of the Frobenius automorphism on the second $l$-adic \'etale cohomology group using Lefschetz fixed point theorem and get the sharp upper bound of the rank of the \Neron-Severi group.
\end{abstract}
\end{document}