%! TEX root = ../main.tex
\documentclass[main]{subfiles}

\begin{document}
\section{Proof of Theorem~\ref{thm:E_{1,s}} and Theorem~\ref{thm:E_{1,s_0}}}

\begin{thm}{(\cite[Exercise 3.9.]{ref:advancedaec})}
    j-invariant が const でなければ non-split
\end{thm}

\begin{proof}[Proof of Theorem~\ref{thm:E_{1,s}}]

    Let $\mathcal{E}_{1,s} \to \mathbb{P}^{1}$ be the elliptic surface with the generic fiber $E_{1,s}$.
    The discriminant of $E_{1,s}$ is
    \begin{equation*}
        \Delta_{E_{1,s}} = 256s^{4} (s + 1)^{4} (s - 1)^{4} (s^{2} + 1)^{4},
    \end{equation*}
    and by Tate's algorithm, we have the following table.
    \begin{table}[ht]
        \centering
        \caption{Singular fibers of $E_{1,s}$}
        \begin{tabular}{|c|c|c|}
            \hline
            Place             & Type  & $m_v$ \\
            \hline
            $s=0$             & $I_4$ & 4     \\
            $s=\pm 1$         & $I_4$ & 4     \\
            $s=\pm \sqrt{-1}$ & $I_4$ & 4     \\
            $s=\infty$        & $I_4$ & 4     \\
            \hline
        \end{tabular}
    \end{table}

    Then $e(\mathcal{E}_{1,s}) = 24$ and by Theorem~\ref{thm:rho}, we have $\rho(\mathcal{E}_{1,s}) \leq 20$.
    By Shioda-Tate formula (Theorem~\ref{thm:shioda}), we have
    \begin{equation*}
        \rank(E_{1,s}) = 0
    \end{equation*}

    As for the torsion subgroup, we have
    \begin{equation*}
        E_{1,s}(\overline{\mathbb{Q}}(s))[2] = \{\mathcal{O}, (0,0), (4s^{2},0),( - (s^{2} - 1)^{2},0)\},
    \end{equation*}
    and we can check by calculation that
    \begin{align}
        2T_1 & = (4s^2,0), \\
        2T_2 & = (0,0).
    \end{align}
    By Theorem~\ref{thm:torsion}, we have $E_{1,s}(\overline{\mathbb{Q}}(s))_ \text{tors} \hookrightarrow (\mathbb{Z} / 4 \mathbb{Z})^{6}$.
    Therefore, we have
    \begin{equation*}
        E_{1,s}(\overline{\mathbb{Q}}(s))_ \text{tors} \cong \mathbb{Z} / 4 \mathbb{Z} \oplus \mathbb{Z} / 4 \mathbb{Z}.
    \end{equation*}
\end{proof}

\begin{thm}{(Mazur's Theorem)}
    \label{thm:mazur}
    Let $E$ be an elliptic curve defined over $\mathbb{Q}$.
    Then the torsion subgroup $E(\mathbb{Q})_{\text{tors}}$ is isomorphic to one of the following groups.
    \begin{align*}
        (1) & \quad \mathbb{Z} / n \mathbb{Z} \quad                                   &  & (1 \leq n \leq 10 \; \text{or}\; n = 12), \\
        (2) & \quad \mathbb{Z} / 2 \mathbb{Z} \oplus \mathbb{Z} / 2n \mathbb{Z} \quad &  & (1 \leq n \leq 4).
    \end{align*}
\end{thm}

\begin{proof}[Proof of Theorem~{\ref{thm:E_{1,s_0}}}]
    By the equation \eqref{eq:specialization} and Mazur's Theorem (Theorem~\ref{thm:mazur}), the only possibility of $E_{1,s_0}(\mathbb{Q})_{\text{tors}}$ is $\mathbb{Z} / 2 \mathbb{Z} \oplus \mathbb{Z} / 4 \mathbb{Z}$ or $\mathbb{Z} / 2 \mathbb{Z} \oplus \mathbb{Z} / 8 \mathbb{Z}$.
    For a point $P=(x,y) \in E_{1,s_0}(\mathbb{Q})$, we can calculate the x coordinate of $2P$ as
    \begin{equation*}
        x(2P) = \frac{16s^{4}(s^{2} - 1)^{4} + 8 s^{2}(s^{2} - 1)^{2} x^{2} + x^{4}}{4 x (x - 4s^{2}) (x + (s^{2} - 1)^{2})}.
    \end{equation*}
    Assume that there is a rational point of order $8$, then there is a point $P=(x,y) \in E_{1,s_0}(\mathbb{Q})$ such that $2P = T_1 = (2s(s+1)^2, 2s(s+1)^2(s^2+1))$.
    Then we have
    \begin{equation*}
        \frac{16s^{4}(s^{2} - 1)^{4} + 8 s^{2}(s^{2} - 1)^{2} x^{2} + x^{4}}{4 x (x - 4s^{2}) (x + (s^{2} - 1)^{2})} = 2s(s+1)^2.
    \end{equation*}
    Put $x' = x - 2s(s+1)^2$, then we have
    \begin{align*}
        x'^4 &= 8s(s^{2} + 1)(s + 1)^{4}(x' + 2s(s^2+1))^{2},\\
        x'^2 &= \pm \sqrt{8 s (s^2 + 1)} (s+1)^{2} (x' + 2s(s^2+1)).
    \end{align*}
    Since $x', s \in \mathbb{Q}$, we have $\sqrt{8 s (s^2 + 1)} \in \mathbb{Q}$.
    Then $(2s, \sqrt{8 s (s^2 + 1)})$ is a rational point on the elliptic curve $y^2 = x^3 + 4x$.
    However, we know that the Mordell-Weil group of $y^2 = x^3 + 4x$ over $\mathbb{Q}$ is
    \begin{equation*}
        \{ \mathcal{O}, (0,0), (2,\pm 4) \}.
    \end{equation*}
    This contradicts the assumption that $s\in \mathbb{Q} \setminus \{0,\pm 1\}$.
\end{proof}
\end{document}
