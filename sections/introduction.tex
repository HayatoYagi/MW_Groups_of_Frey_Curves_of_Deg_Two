%! TEX root = ../main.tex
\documentclass[main]{subfiles}

\begin{document}
\section{Introduction}
Let $(a,b,c)$ be a Pythagorean triple, i.e., $a^{2} + b^{2} = c^{2}$, $abc\neq0$, and $a,b,c \in \mathbb{Z}$, and consider the elliptic curve defined by a Weierstrass equation
\begin{equation*}
    y^{2} = x(x - a^{2})(x + b^{2}),
\end{equation*}
which we call the Frey curve of degree $2$.
There are several examples of this form with rank $0$ and $1$.

We can parameterize the Frey curves of degree $2$ by $s \in \mathbb{Q}$, then the family of the Frey curves of degree $2$ is equivalent to a family of elliptic curves with the following Weierstrass equation
\begin{equation*}
    E_{1,s}: y^{2} = x(x - 4s^{2})(x + (s^{2} - 1)^{2}).
\end{equation*}
We consider $E_{1,s}$ as an elliptic curve over a function field $\overline{\mathbb{Q}}(s)$.
On the relation between an elliptic curve over a function field and its special fibers, the following theorem is known.
\begin{thm}{(Specialization Theorem, \cite[Theorem 11.4.]{ref:advancedaec})}
    \label{thm:specialization}
    Let $E$ be an elliptic curve over a function field $k(C)$ of a smooth projective curve $C$ over an algebraically closed field $k$.
    Let $\pi: \mathcal{E} \to C$ be the \Neron{} model of $E$ and $F_v:=\pi^{-1}(v)$ be the special fiber for $v \in C(k)$.
    Then for all but finitely many $v \in C(k)$, a map
    \begin{equation*}
        E(k(C)) \hookrightarrow F_{v}(k),
    \end{equation*}
    called the specialization map at $v \in C(k)$, is injective
\end{thm}

First, we determine the Mordell-Weil group of $E_{1,s}$.

\begin{thm}
    \label{thm:E_{1,s}}
    The Mordell-Weil group of $E_{1,s}$ over $\overline{\mathbb{Q}}(s)$ satisfies
    \begin{equation*}
        E_{1,s}(\overline{\mathbb{Q}}(s)) \cong \mathbb{Z} / 4 \mathbb{Z} \oplus \mathbb{Z} / 4 \mathbb{Z},
    \end{equation*}
    especially the rank is $0$. The torsion subgroup is generated by
    \begin{align}
        T_1 & := (2s(s+1)^2, 2s(s+1)^2(s^2+1)),                              \\
        T_2 & := (2 \sqrt{-1} s(s^2-1),2 \sqrt{-1} s(s+\sqrt{-1})^2(s^2-1)).
    \end{align}
\end{thm}

\begin{cor}
    \begin{equation*}
        E_{1,s}(\mathbb{Q}(s)) \cong \mathbb{Z} / 2 \mathbb{Z} \oplus \mathbb{Z} / 4 \mathbb{Z}
    \end{equation*}
    is generated by $T_1$ and $2T_2=(0,0)$.
\end{cor}

\begin{rem}
    課題研究では,多項式の解の非存在から背理法を用いて 8-torsion point が存在しないことを示した.
    TODO: specialization theorem は injectivity しか言ってないので, 各 special fiber について 8-torsion point が存在しないことを示した課題研究は上の系より真に強い?
\end{rem}

Inspite of the fact that the generic rank of $E_{1,s}$ is $0$, Yoshida (\cite[Corollary 4.7.]{ref:yoshida}) showed that there are infinitely many $s \in \mathbb{Q}$ such that the Mordell-Weil group of $E_{1,s}$ has positive rank over $\mathbb{Q}$.
However, the infinite family is not explicitly parameterized.
We find the subset of $s \in \overline{{\mathbb{Q}}}$ with the positive rank of the Mordell-Weil group of $E_{1,s}$ parameterized by an rational function of one variable $t \in \overline{\mathbb{Q}}$.
In order to prove it, by the specialization theorem (Theorem~\ref{thm:specialization}), it is enough to find a subfamily of $E_{1,s}$ whose generic rank is $1$.

By substituting $s = \frac{2t}{t^{2} - 3}$ into $E_{1,s}$, we get a new family of elliptic curves
\begin{equation*}
    E_{2,t}: y^{2} = x\left(x - 4\left(\frac{2t}{t^{2} - 3}\right)^{2}\right)\left(x + \left(\left(\frac{2t}{t^{2} - 3}\right)^{2} - 1\right)^{2}\right),
\end{equation*}
which is a subfamily of $E_{1,s}$.

The following is our main result.
\begin{thm}
    \label{thm:E_{2,t}}
    The Mordell-Weil group of $E_{2,t}$ over $\overline{\mathbb{Q}}(t)$ satisfies
    \begin{equation*}
        E_{2,t}(\overline{\mathbb{Q}}(t)) \cong \mathbb{Z} \oplus \mathbb{Z} / 4 \mathbb{Z} \oplus \mathbb{Z} / 4 \mathbb{Z},
    \end{equation*}
    especially the rank is $1$.
    We denote $s = \frac{2t}{t^{2} - 3}$. The torsion subgroup is generated by $T_1$ and $T_2$ in Theorem~\ref{thm:E_{1,s}} and the free part is generated by
    \begin{equation*}
        \left(s^{2} - 1, \sqrt{-1} s(s^{2} - 1) \frac{t^{2} + 3}{t^{2} - 3} \right).
    \end{equation*}
\end{thm}
The important point is that we prove that the generic rank of $E_{2,t}$ is exactly $1$, not only the existence of a point of infinite order.
Our proof is based on the method of Naskręcki in \cite{ref:naskrecki2013}.

\end{document}