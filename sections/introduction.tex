%! TEX root = ../main.tex
\documentclass[main]{subfiles}

\begin{document}
\section{Introduction}

Let $a,b,c$ be positive integers which satisfy the Fermat's equation
\begin{equation*}
    a^{n} + b^{n} = c^{n}.
\end{equation*}
for any integer $n \geq 3$ and consider the elliptic curve defined by the Weierstrass equation
\begin{equation*}
    y^{2} = x(x - a^{n})(x + b^{n}),
\end{equation*}
which is called the Frey curve.
The Frey curve took an important role in the proof of Fermat's Last Theorem by Wiles.
Wiles proved that the Frey curve cannot exist, which implies that the Fermat's equation has no nontrivial solution.

In this paper, we consider elliptic curves in the form of the Frey curves for $n=2$.
In other words, let $(a,b,c)$ be a Pythagorean triple, i.e., $a^{2} + b^{2} = c^{2}$, and consider the elliptic curve defined by the Weierstrass equation
\begin{equation*}
    y^{2} = x(x - a^{2})(x + b^{2}),
\end{equation*}
which we call the Frey curve of degree $2$.
The Frey curves of degree $2$ do exist infinitely unlike for $n \geq 3$.

% 楕円曲線の研究において Mordell-Weil 群は重要な対象であることを書く
For an elliptic curve $E$ defined over a field $K$, the Mordell-Weil group $E(K)$ is a group consisting of all K-rational points on $E$.
The Mordell-Weil group is an important object in the study of elliptic curves.

\begin{thm}{(Mordell's Theorem)}
    \label{thm:mordell}
    Let $E$ be an elliptic curve defined over a number field $K$.
    Then the Mordell-Weil group $E(K)$ is a finitely generated abelian group.
\end{thm}
By the structure theorem of finite abelian groups, the Mordell-Weil group can be decomposed into a free part and a torsion part:
\begin{equation*}
    E(K) \cong \mathbb{Z}^{\oplus r} \oplus E(K)_{\text{tors}}
\end{equation*}
where $r$ is the rank of the Mordell-Weil group and $E(K)_{\text{tors}}$ is the torsion subgroup of $E(K)$.

We can one-parameterize Pythagorean triples by rational numbers and then the Frey curves of degree $2$ are isomorphic to
\begin{equation}
    \label{eq:E_{1,s}}
    E_{1,s}: y^{2} = x(x - 4s^{2})(x + (s^{2} - 1)^{2})
\end{equation}
for some $s \in \mathbb{Q}$.
We consider $E_{1,s}$ as an elliptic curve over a function field $\overline{\mathbb{Q}}(s)$.

For any elliptic curves $E$ over a function field $k(C)$ of a smooth irreducible projective curve $C$ over an algebraically closed field $k$, there is an elliptic surface $\pi: \mathcal{E} \to C$ with the generic fiber $E$ called the \Neron{} model, and we can use some theorems in the theory of surfaces.
The elliptic surface $E/k(C)$ is called the generic fiber of the elliptic surface $\mathcal{E} \to C$.
For $s \in \overline{\mathbb{Q}}$, $\mathcal{E}_s:=\pi^{-1}(s)$ is called the special fiber at $s$.
For all but finitely many $s \in \overline{\mathbb{Q}}$, the special fiber at $s$ is non-singular, which means that is an elliptic curve.

The Mordell's Theorem also holds for elliptic curves over a function field.
\begin{thm}{(\cite[Theorem 6.1.]{ref:advancedaec})}
    \label{thm:mordell_function_field}
    Let $\mathcal{E} \to C$ be an elliptic surface defined over a field $k$ and $E$ be the corresponding elliptic curve over the function field $k(C)$.
    If $\mathcal{E} \to C$ does not split, then the Mordell-Weil group $E(k(C))$ is a finitely generated abelian group.
\end{thm}

On the relation between the Mordell-Weil group of an elliptic curve over a function field and its special fibers, the following theorem is known.
\begin{thm}{(Specialization Theorem, \cite[Theorem 11.4.]{ref:advancedaec})}
    \label{thm:specialization}
    Let $\mathcal{E} \to C$ be a non-split elliptic surface defined over a number field $k$ and $E$ be the corresponding elliptic curve over the function field $k(C)$.
    Let $k'=k$ or $\overline{k}$.
    Then for all but finitely many $s \in C(k')$, a homomorphism
    \begin{equation*}
        E(k(C)) \hookrightarrow \mathcal{E}_{s}(k'),
    \end{equation*}
    called the specialization homomorphism at $s$, is injective.
\end{thm}

\begin{lem}{(\cite[Exercise 3.9.(a)]{ref:advancedaec})}
    Let $\mathcal{E}$ be an elliptic surface over $k$, and let $j_{\mathcal{E}}: C \to \mathbb{P}^1$ be a morphism such that $j_{\mathcal{E}}(v)$ gives the $j$-invariant for any non-singular fibers $\mathcal{E}_v$.
    If $\mathcal{E}$ splits over $k$, then $j_{\mathcal{E}}$ is a constant map.
\end{lem}

\begin{rem}
    We can easily check that all elliptic surfaces in this paper have non-constant $j$-invariants, therefore, they do not split.
    Thus we can apply the Specialization Theorem (Theorem~\ref{thm:specialization}).
\end{rem}

\end{document}