%! TEX root = ../main.tex
\documentclass[main]{subfiles}

\begin{document}
\renewcommand{\abstractname}{論文要旨}
\begin{abstract}
    % 800字程度
    楕円曲線とは,種数 $1$ の滑らかな射影曲線のことである.
    楕円曲線上の点に対して加法を定義することができ,これにより楕円曲線上の点全体は,無限遠点を単位元とするアーベル群となる.
    体 $K$ 上に定義された楕円曲線 $E$ に対して,Mordell-Weil 群 $E(K)$ は $E$ 上の $K$-有理点全体からなる群である.
    Mordell-Weil の定理によると,$K$ がある種の有限条件をみたすとき Mordell-Weil 群は有限生成アーベル群である.
    Mordell-Weil 群は楕円曲線の研究において重要な対象であり,特にそのランクは一般的に決定が難しい.

    本研究では,ピタゴラス数 $(a,b,c) \in \mathbb{Z}^{3}$,すなわち $a^{2} + b^{2} = c^{2}$ を満たす整数 $a,b,c$ に対して,$y^2 = x (x - a^2)(x+b^2)$ で与えられる楕円曲線の族について考察する.
    これは Frey 曲線の $n=2$の場合である.
    ピタゴラス数 $(a,b,c)$ は,整数 $m,n \in \mathbb{Z}$ で $(m,n)=1$ を満たすものにより,$(a,b,c) = (2mn, m^2 - n^2, m^2 + n^2)$ とパラメータ表示される.
    $s = m/n$ とおけば,$2$次 Frey 曲線の族は
    \begin{equation*}
        E_{1,s}: y^{2} = x(x - 4s^{2})(x + (s^{2} - 1)^{2})
    \end{equation*}
    という関数体 $\mathbb{Q}(s)$ 上の楕円曲線として扱うことができる.
    この楕円曲線の $\overline{\mathbb{Q}}(s)$ 上の Mordell-Weil 群のランクは $0$ であることが知られている.
    本研究ではまずこの Mordell-Weil 群の捩れ部分群を完全に決定する.
    またこの楕円曲線の無限部分族で,その Mordell-Weil 群のランクが $1$ であるものを構成する.

    証明には楕円曲面の理論を用いる.
    一般に関数体上の楕円曲線は楕円曲面に対応し, Shioda-Tate の公式は楕円曲面の \Neron-Severi 群のランクと Mordell-Weil 群のランクの関係を与える.

    Tate のアルゴリズムにより楕円曲面の特異ファイバーの型を決定し \Neron-Severi 群のランクの上界を見積もることで,無限部分族のランクが $2$ 以下であることを示すことができる.
    しかしながら,この場合得られる上界は最適ではない.
    Lefschetz の不動点定理を用いて $l$-進エタールコホモロジー に作用する Frobenius 自己同型の特性多項式を計算し,\Neron-Severi 群のランクのより良い上界を得ることで,無限部分族のランクがちょうど $1$ であることを示す.
    % 1002文字
\end{abstract}
\end{document}