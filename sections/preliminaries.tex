%! TEX root = ../main.tex
\documentclass[main]{subfiles}

\begin{document}
\section{Preliminaries}

In order to get the lower bound of the rank of the Mordell-Weil group, finding points of infinite order is enough.
It is quite difficult to get a sharp upper bound of the rank.
The following theorem plays a key role in the proof of the main theorem.
\begin{thm}{(Shioda-Tate formula, \cite[Corollary 5.3]{ref:shioda1990})}
    \label{thm:shioda}
    Let $\mathcal{E} \to C$ be an elliptic surface over a smooth projective curve $C$ over an algebraically closed field $k$.
    Let $R \subset C$ be the set of points where the special fiber of $\mathcal{E}$ is singular.
    For each $v \in R$, let $m_{v}$ be the number of components of the special fiber of $\mathcal{E}$ at $v$.
    Let $\rho(\mathcal{E})$ denote the rank of the \Neron-Severi group of $\mathcal{E}$.
    Then, we have
    \begin{equation*}
        \rho (\mathcal{E}) = 2 + \sum_{v \in R} (m_{v} - 1) + \rank(E(k(C))).
    \end{equation*}
\end{thm}

We have the following theorem giving an upper bound of the Picard number $\rho(\mathcal{E})$.

\begin{thm}{(\cite[Twierdzenie 2.2.9, 2.2.10, 2.2.19]{ref:naskreckiphd})}
    Let $\mathcal{E}$ and $R$ be as in Theorem~\ref{thm:shioda}.
    Let $\chi(\mathcal{E})$ be the arithmetic genus of $\mathcal{E}$, $e(\mathcal{E}_v)$ be the local Euler number of the special fiber at $v$, and $g(C)$ be the genus of $C$. \label{thm:rho}
    Assume $\ch k = 0$.
    Then
    \begin{equation*}
        12 \chi(\mathcal{E}) = e(\mathcal{E}) := \sum_{v \in R} e(\mathcal{E}_{v}),
    \end{equation*}
    \begin{equation*}
        \rho(\mathcal{E}) \leq 10 \chi(\mathcal{E}) + 2g(C).
    \end{equation*}
    If $\chi(\mathcal{E}) = 1$, the elliptic surface $\mathcal{E}$ is called rational.
    If $\chi(\mathcal{E}) = 2$, the elliptic surface $\mathcal{E}$ is called an elliptic K3 surface.
\end{thm}

We can compute types of each special fibers denoted by Kodaira symbols by Tate's algorithm (\cite[IV \S 9]{ref:advancedaec}).
The following table shows the correspondence between Kodaira symbols and the values of $m_v$ and $e(\mathcal{E}_v)$ appearing in the two theorems above.

\begin{table}[H]
    \centering
    \caption{Kodaira symbols (\cite[pp.136-137 付録2]{ref:shioda1993})}
    \begin{tabular}{|c|c|c|}
        \hline
        Kodaira symbol & $m_v$ & $e(\mathcal{E}_v)$ \\
        \hline
        $\mathrm{I}_n$   & $n$   & $n$                \\
        $\mathrm{II}$    & $1$   & $2$                \\
        $\mathrm{III}$   & $2$   & $3$                \\
        $\mathrm{IV}$    & $3$   & $4$                \\
        $\mathrm{I}_n^*$ & $n+5$ & $n+6$              \\
        $\mathrm{II}^*$  & $9$   & $10$               \\
        $\mathrm{III}^*$ & $8$   & $9$                \\
        $\mathrm{IV}^*$  & $7$   & $8$                \\
        \hline
    \end{tabular}
    \label{tab:kodaira}
\end{table}

For the torsion subgroup of the Mordell-Weil group, we have the following theorem.
\begin{thm}{(\cite[Lem.3.5]{ref:naskrecki2013})}
    \label{thm:torsion}
    Let $\mathcal{E}$ and $R$ be as in Theorem~\ref{thm:shioda}.
    Let $E$ be the generic fiber of $\mathcal{E}$.
    Then we have
    \begin{equation*}
        E(\overline{\mathbb{Q}}(s))_{\text{tors}} \hookrightarrow \prod_{v \in R} G(\mathcal{E}_{v}),
    \end{equation*}
    where $G(\mathcal{E}_{v})$ is the group generated by all simple components of the fiber at $v$.
    If $\mathcal{E}_{v}$ is of type $\mathrm{I}_n$ in Kodaira symbol, then $G(\mathcal{E}_{v}) \cong \mathbb{Z} / n \mathbb{Z}$.
\end{thm}

\end{document}
