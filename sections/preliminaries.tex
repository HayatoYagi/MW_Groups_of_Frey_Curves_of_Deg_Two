%! TEX root = ../main.tex
\documentclass[main]{subfiles}

\begin{document}
\chapter{Preliminaries}

In order to get the lower bound of the rank of the Mordell-Weil group, finding generators is enough.
It is more difficult to get the upper bound of the rank.
The following theorem behaves a key role in the proof of the main theorem.
\begin{thm}{(Shioda-Tate formula, \cite[Corollary 5.3]{ref:shioda1990})}
    \label{thm:shioda}
    Let $C$ be a smooth irreducible projective curve over an algebraically closed field $k$ and $E$ an elliptic curve over a function field $k(C)$.
    Let $\mathcal{E} \to C$ be the \Neron{} model of $E$.
    Let $R \subset C$ be the set of points where the special fiber of $\mathcal{E}$ is singular.
    For each $v \in R$, let $m_{v}$ be the number of components of the special fiber of $\mathcal{E}$ at $v$.
    Let $\rho(\mathcal{E})$ denote the rank of the \Neron-Severi group of $\mathcal{E}$.
    Then, we have
    \begin{equation}
        \rho (\mathcal{E}) = 2 + \sum_{v \in R} (m_{v} - 1) + \rank(E(k(C))).
    \end{equation}
\end{thm}

We can calculate $R$ and $m_{v}$ by Tate's algorithm, but it is still difficult to determine $\rho(\mathcal{E})$.
We have the following theorem to get the upper bound of $\rho(\mathcal{E})$.

\begin{thm}
    \label{thm:rho}
    % \begin{equation}
        % 12 \chi = \sum e(F_{v})
        %\end{equation}
    % \begin{equation}
        % \rho(\mathcal{E}) \leq 10 \chi + 2g
        %\end{equation}
    \begin{align}
        \rho(\mathcal{E}) \leq \frac{5}{6} e(\tilde{S}) + 2, \\
        e(\tilde{S}) := \sum_{v \in R} e(F_{v}).
    \end{align}
    where $e(\tilde{S})$ is the Euler number, $e(F_{v})$ is the local Euler number of the special fiber of $\mathcal{E}$ at $v$ for each $v \in R$ and
    \begin{align}
        e(F_{v}) = \begin{cases}
                       m_v     & \text{if the fiber has multiplicative reduction}, \\
                       m_v + 1 & \text{if the fiber has additive reduction}.       \\
                   \end{cases}
    \end{align}
\end{thm}
\begin{proof}
    TODO: Naskrencki の PhD の 2.2.19(ii), 2.2.9, 2.2.10 (やその引用元)を引用する
\end{proof}


However, Theorem~\ref{thm:rho} is still not enough to get the upper bound of the ranks of the Mordell-weil groups in our case.

TODO: \'etale cohomology を使うことを説明する.


\begin{dfn}
    Let $C$ be a smooth curve over an algebraically closed field $k$.
    Let $E$ be an elliptic curve over a function field $k(C)$ given by the Weierstrass equation
    \begin{equation}
        E: y^{2} = x^{3} + a_{2} x^2 + a_{4} x + a_{6}
    \end{equation}
    where $a_{2}, a_{4}, a_{6} \in k(C)$.
    For a fixed $u \in k(C)^*$, we denote
    \begin{equation}
        E^{(u)}: u y^{2} = x^{3} + a_{2} x^2 + a_{4} x + a_{6}
    \end{equation}
    to be the quadratic twist of $E$ by $u$.
\end{dfn}

The following theorem is used to reduce the order of coefficients in the Weierstrass equation and make the computation feasible.

\begin{thm}{(\cite[Exercise 10.16]{ref:aec})}
    Let $E$ be an elliptic curve over a function field $k(C)$ and $u \in k(C)^*$.
    Then, the following equation holds
    \begin{equation}
        \rank E(k(C)) + \rank E^{(u)}(k(C)) = \rank E(k(C)(\sqrt{u})).
    \end{equation}
\end{thm}

\begin{thm}{TODO: どこに書くか検討}
    \begin{equation}
        E(\overline{\mathbb{Q}}(s))_{\text{tors}} \hookrightarrow \prod_{v \in R} G(F_{v})
    \end{equation}
    where $G(F_{v})$ is the group generated by all simple components of the fiber at $v$.
    If $F_v$ is of type $I_n$ in Kodaira notation, then $G(F_{v}) \cong \mathbb{Z} / n \mathbb{Z}$.
\end{thm}
\begin{proof}
    TODO
\end{proof}

\end{document}
