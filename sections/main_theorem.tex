%! TEX root = ../main.tex
\documentclass[main]{subfiles}

\begin{document}

\section{Main theorem}
First, for the elliptic curve $E_{1,s}$ defined in \eqref{eq:E_{1,s}}, we determine the Mordell-Weil group over $\overline{\mathbb{Q}}(s)$.

\begin{thm}
    \label{thm:E_{1,s}}
    The Mordell-Weil group of $E_{1,s}$ over $\overline{\mathbb{Q}}(s)$ satisfies
    \begin{equation*}
        E_{1,s}(\overline{\mathbb{Q}}(s)) \cong \mathbb{Z} / 4 \mathbb{Z} \oplus \mathbb{Z} / 4 \mathbb{Z},
    \end{equation*}
    especially the rank is $0$. The torsion subgroup is generated by
    \begin{align*}
        T_1 & := (2s(s+1)^2, 2s(s+1)^2(s^2+1)),                               \\
        T_2 & := (2 \sqrt{-1} s(s^2-1),2 \sqrt{-1} s(s+\sqrt{-1})^2(s^2-1)).
    \end{align*}
\end{thm}

\begin{cor}
    \label{cor:E_{1,s}}
    \begin{equation*}
        E_{1,s}(\mathbb{Q}(s)) \cong \mathbb{Z} / 2 \mathbb{Z} \oplus \mathbb{Z} / 4 \mathbb{Z}
    \end{equation*}
    is generated by $T_1$ and $2T_2=(0,0)$.
\end{cor}

By Corollary~\ref{cor:E_{1,s}} and the specialization theorem (Theorem~\ref{thm:specialization}), we have
\begin{equation*}
    \mathbb{Z} / 2 \mathbb{Z} \oplus \mathbb{Z} / 4 \mathbb{Z} \hookrightarrow E_{1,s_{0}}(\mathbb{Q})
\end{equation*}
for all but finitely many $s_0 \in \mathbb{Q}$.
Actually, we can prove that the image of the specialization homomorphism is $E_{1,s_0}(\mathbb{Q})_{\text{tors}}$ for all $s_0 \in \mathbb{Q}$ unless $E_{1,s_0}$ is singular.

\begin{thm}
    \label{thm:E_{1,s_0}}
    For any $s_0 \in \mathbb{Q} \setminus \{0,\pm 1 \}$, $E_{1,s_0}$ is non-singular and the torsion subgroup satisfies
    \begin{equation*}
        E_{1,s_{0}}(\mathbb{Q})_{\text{tors}} \cong \mathbb{Z} / 2 \mathbb{Z} \oplus \mathbb{Z} / 4 \mathbb{Z}.
    \end{equation*}
\end{thm}

Inspite of the fact that the generic rank of $E_{1,s}$ is $0$, Yoshida (\cite[Corollary 4.7.]{ref:yoshida}) showed that there are infinitely many $s \in \mathbb{Q}$ such that the Mordell-Weil group of $E_{1,s}$ has positive rank over $\mathbb{Q}$.
However, the infinite family is not explicitly parameterized.
We find the subset of $s \in \overline{{\mathbb{Q}}}$ with the positive rank of the Mordell-Weil group of $E_{1,s}$ parameterized by a rational function of one variable $t \in \overline{\mathbb{Q}}$.
In order to prove it, by the specialization theorem (Theorem~\ref{thm:specialization}), it is enough to find a subfamily of $E_{1,s}$ whose generic rank is $1$.

By substituting $s = \frac{2t}{t^{2} - 3}$ into $E_{1,s}$, we get a new family of elliptic curves
\begin{equation*}
    E_{2,t}: y^{2} = x \left(x - 4 \left(\frac{2t}{t^{2} - 3} \right)^{2} \right) \left(x + \left(\left(\frac{2t}{t^{2} - 3} \right)^{2} - 1 \right)^{2} \right),
\end{equation*}
which is a subfamily of $E_{1,s}$.

The following is our main result.
\begin{thm}
    \label{thm:E_{2,t}}
    The Mordell-Weil group of $E_{2,t}$ over $\overline{\mathbb{Q}}(t)$ satisfies
    \begin{equation*}
        E_{2,t}(\overline{\mathbb{Q}}(t)) \cong \mathbb{Z} \oplus \mathbb{Z} / 4 \mathbb{Z} \oplus \mathbb{Z} / 4 \mathbb{Z},
    \end{equation*}
    especially the rank is $1$.
    % We denote $s = \frac{2t}{t^{2} - 3}$. The torsion subgroup is generated by $T_1$ and $T_2$ in Theorem~\ref{thm:E_{1,s}} and the free part is generated by
    % \begin{equation*}
    %     \left(s^{2} - 1, \sqrt{-1} s(s^{2} - 1) \frac{t^{2} + 3}{t^{2} - 3} \right).
    % \end{equation*}
    The torsion subgroup is generated by $T_1$ and $T_2$ in Theorem~\ref{thm:E_{1,s}} with $s = \frac{2t}{t^{2} - 3}$.
\end{thm}
The important point is that we prove that the generic rank of $E_{2,t}$ is exactly $1$, not only the existence of a point of infinite order.
We use the Tate's algorithm, the Shioda-Tate formula, and the Lefschetz fixed point theorem, which we will explain in the following sections, to prove Theorem~\ref{thm:E_{2,t}}.
Our proof is based on the method of Naskręcki in \cite{ref:naskrecki2013}.

\begin{rem}
    All points in $E_{2,t}(\overline{\mathbb{Q}}(t))$ are defined over $\mathbb{Q}(\sqrt{-1})(t)$.
    Thus it follows that
    \begin{equation*}
        E_{2,t}(\mathbb{Q}({\sqrt{-1}})(t)) = E_{2,t}(\overline{\mathbb{Q}}(t)) \cong \mathbb{Z} \oplus \mathbb{Z} / 4 \mathbb{Z} \oplus \mathbb{Z} / 4 \mathbb{Z},
    \end{equation*}
    especially the generic rank of $E_{2,t}$ over $\mathbb{Q}(\sqrt{-1})(t)$ is $1$.
    However, the points of infinite order are not defined over $\mathbb{Q}(t)$.
    Thus the generic rank of $E_{2,t}$ over $\mathbb{Q}(t)$ is $0$.
    If we consider the elliptic curve with Weierstrass equation
    \begin{equation*}
        E_{2,t}^{(-1)}: -y^{2} = x \left(x - 4 \left(\frac{2t}{t^{2} - 3} \right)^{2} \right) \left(x + \left(\left(\frac{2t}{t^{2} - 3} \right)^{2} - 1 \right)^{2} \right),
    \end{equation*}
    then the generic rank over $\mathbb{Q}(t)$ is $1$.
\end{rem}

\end{document}
