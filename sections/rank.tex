%! TEX root = ../main.tex
\documentclass[main]{subfiles}

\begin{document}
\section{The Generic Rank of \texorpdfstring{$E_{2,t}$}{E2,t}}

In order to prove Theorem~\ref{thm:E_{2,t}}, Theorem~\ref{thm:rho} is not enough to get the sharp upper bound of the ranks of the \Neron-Severi group.
Actually, the discriminant of $E_{2,t}$ is
\begin{equation*}
    \Delta_{E_{2,t}} = 4096t^{4}(t - 1)^{4}(t + 1)^{4}(t - 3)^{4}(t + 3)^{4}(t^{2} - 3)^{4}(t^{4} - 2t^{2} + 9)^{4},
\end{equation*}
% the $j$-invariant is
% \begin{equation*}
%     j_{E_{2,t}} = \frac{(s^{8} - 4 s^{7} - 4 s^{6} + 52 s^{5} + 22 s^{4} - 156 s^{3} - 36 s^{2} + 108 s + 81)^{3} (s^{8} + 4 s^{7} - 4 s^{6} - 52 s^{5} + 22 s^{4} + 156 s^{3} - 36 s^{2} - 108 s + 81)^{3}}{t^{4}(t - 1)^{4}(t + 1)^{4}(t - 3)^{4}(t + 3)^{4}(t^{2} - 3)^{4}(t^{4} - 2t^{2} + 9)^{4}}
% \end{equation*}
and the types of the singular fibers of $E_{2,t}$ are calculated as in Table~\ref{tab:E_{2,t}} by Tate's algorithm.
Then in a similar way to the proof of Theorem~\ref{thm:E_{1,s}}, we get $\rank E_{2,t}(\overline{\mathbb{Q}}(t)) \leq 2$.
We will see later that this bound is not sharp.
\begin{table}[H]
    \centering
    \caption{Singular fibers of $E_{2,t}$}
    \begin{tabular}{|c|c|c|c|}
        \hline
        Place            & Type  & $m_v$ & $e$ \\
        \hline
        $t=0$            & $I_4$ & 4     & 4   \\
        $t=\pm 1$        & $I_4$ & 4     & 4   \\
        $t=\pm 3$        & $I_4$ & 4     & 4   \\
        $t=\pm \sqrt{3}$ & $I_4$ & 4     & 4   \\
        $t^4-2t^2+9=0$   & $I_4$ & 4     & 4   \\
        $t=\infty$       & $I_4$ & 4     & 4   \\
        \hline
    \end{tabular}
    \label{tab:E_{2,t}}
\end{table}

\begin{lem}
    \begin{equation*}
        E_{2,t}(\overline{\mathbb{Q}}(t))_{\text{tors}} \cong \mathbb{Z} / 4 \mathbb{Z} \oplus \mathbb{Z} / 4 \mathbb{Z}.
    \end{equation*}
\end{lem}
\begin{proof}
    Since $E_{2,t}$ is given by substituting $s = \frac{2t}{t^{2} - 3}$ into $E_{1,s}$,
    \begin{equation*}
        E_{1,s}(\overline{\mathbb{Q}}(s))_{\text{tors}} \subset E_{2,t}(\overline{\mathbb{Q}}(t))_{\text{tors}}.
    \end{equation*}
    By Theorem~\ref{thm:torsion}, we have
    \begin{equation*}
        E_{2,t}(\overline{\mathbb{Q}}(t))_{\text{tors}} \hookrightarrow (\mathbb{Z} / 4 \mathbb{Z})^{12}.
    \end{equation*}
    Since $E_{1,s}(\overline{\mathbb{Q}}(s))_{\text{tors}} \cong (\mathbb{Z} / 4 \mathbb{Z})^2$ by Theorem~\ref{thm:E_{1,s}}, we have
    \begin{equation*}
        (\mathbb{Z} / 4 \mathbb{Z})^2 \cong E_{1,s}(\overline{\mathbb{Q}}(s))_{\text{tors}} \subset E_{2,t}(\overline{\mathbb{Q}}(t))_{\text{tors}} \subset (\mathbb{Z} / 4 \mathbb{Z})^{12}.
    \end{equation*}
    There are no more points of order $2$ in $E_{2,t}(\overline{\mathbb{Q}}(t))$ than $E_{1,s}(\overline{\mathbb{Q}}(s))$.
    Therefore, we have
    \begin{equation*}
        E_{2,t}(\overline{\mathbb{Q}}(t))_{\text{tors}} \cong \mathbb{Z} / 4 \mathbb{Z} \oplus \mathbb{Z} / 4 \mathbb{Z}.
    \end{equation*}
\end{proof}

\begin{lem}
    \begin{equation*}
        \rank E_{2,t}(\overline{\mathbb{Q}}(t)) \geq 1
    \end{equation*}
\end{lem}
\begin{proof}
    We have a point of infinite order
    \begin{equation*}
        \left(s^{2} - 1, \sqrt{-1} s(s^{2} - 1) \frac{t^{2} + 3}{t^{2} - 3} \right) \in E_{2,t}(\overline{\mathbb{Q}}(t)) \setminus E_{2,t}(\overline{\mathbb{Q}}(t))_{\text{tors}}.
    \end{equation*}
\end{proof}
Now, our goal is to show the rank of $E_{2,t}(\overline{\mathbb{Q}}(t))$ is $\leq 1$.
We use another method to estimate an upper bound of the rank of \Neron-Severi group, which we will explain in Section~\ref{sec:reduction}.
Beforehand, we express the rank of $E_{2,t}(\overline{\mathbb{Q}}(t))$ in terms of ranks of elliptic curves with lower order coefficients in the Weierstrass equations to make the later computation feasible.

\begin{dfn}
    Let $C$ be a smooth curve over an algebraically closed field $k$.
    Let $E$ be an elliptic curve over a function field $k(C)$ given by the Weierstrass equation
    \begin{equation*}
        E: y^{2} = x^{3} + a_{2} x^{2} + a_{4} x + a_{6}
    \end{equation*}
    where $a_{2}, a_{4}, a_{6} \in k(C)$.
    For a fixed $u \in k(C)^*$, we denote
    \begin{equation*}
        E^{(u)}: u y^{2} = x^{3} + a_{2} x^{2} + a_{4} x + a_{6}
    \end{equation*}
    to be the quadratic twist of $E$ by $u$.
\end{dfn}

\begin{prop}{(\cite[Exercise 10.16]{ref:aec})}
    \label{prop:twist}
    Let $E$ be an elliptic curve over a function field $k(C)$ and $u \in k(C)^*$.
    Then, the following equation holds
    \begin{equation*}
        \rank E(k(C)(\sqrt{u})) = \rank E(k(C)) + \rank E^{(u)}(k(C)).
    \end{equation*}
\end{prop}
We omit the proof of Proposition~\ref{prop:twist}.

\begin{thm}
    Let
    \begin{equation*}
        E_{0,u}: y^{2} = x(x - 4u)(x + (u - 1)^{2})
    \end{equation*}
    be an elliptic curve over $\overline{\mathbb{Q}}(u)$.
    Then, we have
    \begin{align}
        \rank E_{2,t}(\overline{\mathbb{Q}}(t))                & = \rank E_{1,s}(\overline{\mathbb{Q}}(s)) + \rank E_{1,s}^{(1 + 3s^{2})}(\overline{\mathbb{Q}}(s)), \label{eq:twist1}           \\
        \rank E_{1,s}^{(1 + 3s^{2})}(\overline{\mathbb{Q}}(s)) & = \rank E_{0,u}^{(1 + 3u)}(\overline{\mathbb{Q}}(u)) + \rank E_{0,u}^{(u(1 + 3u))}(\overline{\mathbb{Q}}(u)). \label{eq:twist2}
    \end{align}
    Therefore, we have
    \begin{equation}
        \label{eq:rankdecomposition}
        \begin{split}
            \rank E_{2,t}(\overline{\mathbb{Q}}(t)) & = \rank E_{1,s}(\overline{\mathbb{Q}}(s))                \\
                                                    & + \rank E_{0,u}^{(1 + 3u)}(\overline{\mathbb{Q}}(u))     \\
                                                    & + \rank E_{0,u}^{(u(1 + 3u))}(\overline{\mathbb{Q}}(u)).
        \end{split}
    \end{equation}
\end{thm}
\begin{proof}
    Since solving $s = \frac{2t}{t^{2} - 3}$ for $t$ yields $t = \frac{1 \pm \sqrt{1 + 3s^{2}}}{s}$, we have
    \begin{equation*}
        E_{2,t}(\overline{\mathbb{Q}}(t)) = E_{1,s}(\overline{\mathbb{Q}}(s)(\sqrt{1 + 3s^{2}})).
    \end{equation*}
    By Proposition~\ref{prop:twist}, we get \eqref{eq:twist1}.
    Similarly, $E_{1,s}$ is obtained by substituting $u = s^{2}$ into $E_{0,u}$, so we have
    \begin{equation*}
        E_{1,s}^{(1 + 3s^{2})}(\overline{\mathbb{Q}}(s)) = E_{0,u}^{(1 + 3u)}(\overline{\mathbb{Q}}(u)(\sqrt{u})),
    \end{equation*}
    then we get \eqref{eq:twist2}.
\end{proof}

We already know that the rank of $E_{1,s}(\overline{\mathbb{Q}}(s))$ is $0$.
The rank of $E_{0,u}^{(u(1 + 3u))}(\overline{\mathbb{Q}}(u))$ in the equation \eqref{eq:rankdecomposition} can also be calculated easily as follows.

\begin{thm}
    \begin{equation*}
        \rank E_{0,u}^{(u(1 + 3u))}(\overline{\mathbb{Q}}(u)) = 1
    \end{equation*}
\end{thm}
\begin{proof}
    We have a point of infinite order
    \begin{equation*}
        (u - 1, \sqrt{-1}(u - 1)) \in E_{0,u}^{(u(1 + 3u))}(\overline{\mathbb{Q}}(u))
    \end{equation*}
    and thus the rank is positive.
    The discriminant of $E_{0,u}^{(u(1 + 3u))}$ is
    \begin{equation*}
        \Delta_{E_{0,u}^{(u(1 + 3u))}} = 256u^{8}(u - 1)^{4}(u + 1)^{4}(3u + 1)^{6},
    \end{equation*}
    and the types of the singular fibers of $E_{0,u}^{(u(1 + 3u))}$ are calculated as in Table~\ref{tab:E_{0,u}^{(u(1 + 3u))}} by Tate's algorithm.
    \begin{table}[H]
        \centering
        \caption{Singular fibers of $E_{0,u}^{(u(1 + 3u))}$}
        \begin{tabular}{|c|c|c|c|}
            \hline
            Place            & Type    & $m_v$ & $e$ \\
            \hline
            $u=0$            & $I_2^*$ & 7     & 8   \\
            $u=\pm 1$        & $I_4$   & 4     & 4   \\
            $u=-\frac{1}{3}$ & $I_0^*$ & 5     & 6   \\
            $u=\infty$       & $I_2$   & 2     & 2   \\
            \hline
        \end{tabular}
        \label{tab:E_{0,u}^{(u(1 + 3u))}}
    \end{table}
    Using Theorem~\ref{thm:shioda} and Theorem~\ref{thm:rho}, we have 
    \begin{equation*}
        e(\mathcal{E}_{0,u}^{(u(1 + 3u))}) = 8 + 4 \times 2 + 6 + 2 = 24
    \end{equation*}
    and
    \begin{align*}
        \rank E_{0,u}^{(u(1 + 3u))}(\overline{\mathbb{Q}}(u)) & \leq 20 - (2 + (7 - 1) + (4 - 1) \times 2 + (5 - 1) + (2 - 1)) \\
                                                              &= 1.
    \end{align*}
\end{proof}

The remaining task is to calculate the rank of $E_{0,u}^{(1 + 3u)}(\overline{\mathbb{Q}}(u))$.
Theorem~\ref{thm:rho} gives the rank is $\leq 1$, which is not sharp.
We will show the rank is $0$ in the next section.

\end{document}
